% ****** Start of file apssamp.tex ******
%
%   This file is part of the APS files in the REVTeX 4.2 distribution.
%   Version 4.2a of REVTeX, December 2014
%
%   Copyright (c) 2014 The American Physical Society.
%
%   See the REVTeX 4 README file for restrictions and more information.
%
% TeX'ing this file requires that you have AMS-LaTeX 2.0 installed
% as well as the rest of the prerequisites for REVTeX 4.2
%
% See the REVTeX 4 README file
% It also requires running BibTeX. The commands are as follows:
%
%  1)  latex apssamp.tex
%  2)  bibtex apssamp
%  3)  latex apssamp.tex
%  4)  latex apssamp.tex
%
\documentclass[%
 reprint,
%superscriptaddress,
%groupedaddress,
%unsortedaddress,
%runinaddress,
%frontmatterverbose, 
%preprint,
%preprintnumbers,
%nofootinbib,
%nobibnotes,
%bibnotes,
 amsmath,amssymb,
 aps,
%pra,
%prb,
%rmp,
%prstab,
%prstper,
%floatfix,
]{revtex4-2}
\usepackage{kotex}
\usepackage{graphicx}% Include figure files
\usepackage{dcolumn}% Align table columns on decimal point
\usepackage{bm}% bold math
%\usepackage{hyperref}% add hypertext capabilities
%\usepackage[mathlines]{lineno}% Enable numbering of text and display math
%\linenumbers\relax % Commence numbering lines

%\usepackage[showframe,%Uncomment any one of the following lines to test 
%%scale=0.7, marginratio={1:1, 2:3}, ignoreall,% default settings
%%text={7in,10in},centering,
%%margin=1.5in,
%%total={6.5in,8.75in}, top=1.2in, left=0.9in, includefoot,
%%height=10in,a5paper,hmargin={3cm,0.8in},
%]{geometry}

\def\rcurs{{\mbox{$\resizebox{.16in}{.08in}{\includegraphics{ScriptR}}$}}}
\def\brcurs{{\mbox{$\resizebox{.16in}{.08in}{\includegraphics{BoldR}}$}}}
\def\hrcurs{{\mbox{$\hat \brcurs$}}}

\begin{document}


\title{화학전지 실험 결과보고서}

\author{서울대학교 전기정보공학부 2018-12432 박정현}
 \email{alexist@snu.ac.kr}
\date{\today}% It is always \today, today,
             %  but any date may be explicitly specified

\begin{abstract}
abdcd
\end{abstract}

%\keywords{Suggested keywords}%Use showkeys class option if keyword
                              %display desired
\maketitle

%\tableofcontents

\section{\label{sec:level1}Assignment}
\subsection{\label{sec:level2}Problem1}
\subsection{\label{sec:level2}Problem2}


\section{\label{sec:level1}Introduction}
\subsection{\label{sec:level2}$NaOH$표준화}
$KHP$와 $NaOH$ 화학 반응식은 아래와 같다.
\begin{align}
	NaOH + KHC_{8}H_{4}O_{4} &\rightarrow KNaC_{8}H_{4}O_{4} + H_{2}O
\end{align}
본 실험에서 사용된 표준 용액은 $KHC_{8}H_{4}O_{4}$ $6mM$이다. 이 때 당량점에서 아래의 식이 성립한다. 이 때 위의 반응식에서 각각 하나의 $H^{+}$, $OH^{-}$가 반응하게 되므로 $n_{KHP}$, $n_{NaOH}$의 값은 $1$이다.
\begin{align}
	n_{KHP}M_{KHP}V_{KHP} &= n_{NaOH}M_{NaOH}V_{NaOH}\\
	M_{NaOH} &= \frac{M_{KHP}V_{KHP}}{V_{NaOH}}
\end{align}
이때 증류수 바탕적정에 투입된 $NaOH$의 양을 뺀 뒤 사용된 위의 식을 이용해 $NaOH$의 농도를 계산할 수 있다.

\section{\label{sec:level1}Data and Results}
\subsection{\label{sec:level2}$NaOH$표준화}
$KHP$용액을 적정한 결과는 아래와 같다.
\begin{table}[]
\begin{tabular}{c|c} \hline \hline
$KHP$ 부피[$mL$] & $NaOH$ 부피[$mL$] \\ \hline
$0.1$ & $3.38 - 3.18 - 0.04 = 0.16$  \\ \hline
$0.5$ & $4.30 - 3.50 - 0.04 = 0.76$  \\ \hline
$0.5$ & $5.11-4.35 - 0.04 = 0.72$ \\  \hline \hline 
\end{tabular}
\caption{\label{tab:chemreaction}측정된 화학 반응여부}
\end{table}

\section{\label{sec:level1}Reference}
[1] 김희준, \textit{일반화학 실험}(자유아카데미, 2016)\\

[2] D.W. Oxtoby, H.P. Gillis, and L. Butler, \textit{Principles of Modern Chemistry} (Brooks/Cole, Australia, 2020).

\end{document}
%
% ****** End of file apssamp.tex ******
