% ****** Start of file apssamp.tex ******
%
%   This file is part of the APS files in the REVTeX 4.2 distribution.
%   Version 4.2a of REVTeX, December 2014
%
%   Copyright (c) 2014 The American Physical Society.
%
%   See the REVTeX 4 README file for restrictions and more information.
%
% TeX'ing this file requires that you have AMS-LaTeX 2.0 installed
% as well as the rest of the prerequisites for REVTeX 4.2
%
% See the REVTeX 4 README file
% It also requires running BibTeX. The commands are as follows:
%
%  1)  latex apssamp.tex
%  2)  bibtex apssamp
%  3)  latex apssamp.tex
%  4)  latex apssamp.tex
%
\documentclass[%
 reprint,
%superscriptaddress,
%groupedaddress,
%unsortedaddress,
%runinaddress,
%frontmatterverbose, 
%preprint,
%preprintnumbers,
%nofootinbib,
%nobibnotes,
%bibnotes,
 amsmath,amssymb,
 aps,
%pra,
%prb,
%rmp,
%prstab,
%prstper,
%floatfix,
]{revtex4-2}
\usepackage{kotex}
\usepackage{graphicx}% Include figure files
\usepackage{dcolumn}% Align table columns on decimal point
\usepackage{bm}% bold math
%\usepackage{hyperref}% add hypertext capabilities
%\usepackage[mathlines]{lineno}% Enable numbering of text and display math
%\linenumbers\relax % Commence numbering lines

%\usepackage[showframe,%Uncomment any one of the following lines to test 
%%scale=0.7, marginratio={1:1, 2:3}, ignoreall,% default settings
%%text={7in,10in},centering,
%%margin=1.5in,
%%total={6.5in,8.75in}, top=1.2in, left=0.9in, includefoot,
%%height=10in,a5paper,hmargin={3cm,0.8in},
%]{geometry}

\def\rcurs{{\mbox{$\resizebox{.16in}{.08in}{\includegraphics{ScriptR}}$}}}
\def\brcurs{{\mbox{$\resizebox{.16in}{.08in}{\includegraphics{BoldR}}$}}}
\def\hrcurs{{\mbox{$\hat \brcurs$}}}

\begin{document}


\title{화학전지 실험 결과보고서}

\author{서울대학교 전기정보공학부 2018-12432 박정현}
 \email{alexist@snu.ac.kr}
\date{\today}% It is always \today, today,
             %  but any date may be explicitly specified

\begin{abstract}
본 실험에서는 여러 물질의 전기전도도를 확인하여 전류에 대한 이해도를 높인다. 또한 여러 금속들의 반응성을 비교해 전기화학적 서열을 실험적으로 확인하였다. 또한 아연, 구리 등과 같은 금속돌의 다니엘 전지를 만들고 전압을  측정해 네른스트식을 이해하고 검증하였다. 실험에서 나타난 대부분의 오차는 금속산화물에 의한 것으로 결론지었으며 해당 산화물을 완전히 제거하면 더 높은 정확도의 실험 결과를 얻을 것으로 결론지었다.
\end{abstract}

%\keywords{Suggested keywords}%Use showkeys class option if keyword
                              %display desired
\maketitle

%\tableofcontents

\section{\label{sec:level1}Assignment}
\subsection{\label{sec:level2}Problem1}
asdf

\section{\label{sec:level1}Reference}
[1] 김희준, \textit{일반화학 실험}(자유아카데미, 2016)\\

[2] D.W. Oxtoby, H.P. Gillis, and L. Butler, \textit{Principles of Modern Chemistry} (Brooks/Cole, Australia, 2020).

\end{document}
%
% ****** End of file apssamp.tex ******
