% ****** Start of file apssamp.tex ******
%
%   This file is part of the APS files in the REVTeX 4.2 distribution.
%   Version 4.2a of REVTeX, December 2014
%
%   Copyright (c) 2014 The American Physical Society.
%
%   See the REVTeX 4 README file for restrictions and more information.
%
% TeX'ing this file requires that you have AMS-LaTeX 2.0 installed
% as well as the rest of the prerequisites for REVTeX 4.2
%
% See the REVTeX 4 README file
% It also requires running BibTeX. The commands are as follows:
%
%  1)  latex apssamp.tex
%  2)  bibtex apssamp
%  3)  latex apssamp.tex
%  4)  latex apssamp.tex
%
\documentclass[%
 reprint,
%superscriptaddress,
%groupedaddress,
%unsortedaddress,
%runinaddress,
%frontmatterverbose, 
%preprint,
%preprintnumbers,
%nofootinbib,
%nobibnotes,
%bibnotes,
 amsmath,amssymb,
 aps,
%pra,
%prb,
%rmp,
%prstab,
%prstper,
%floatfix,
]{revtex4-2}
\usepackage{kotex}
\usepackage{graphicx}% Include figure files
\usepackage{dcolumn}% Align table columns on decimal point
\usepackage{bm}% bold math
%\usepackage{hyperref}% add hypertext capabilities
%\usepackage[mathlines]{lineno}% Enable numbering of text and display math
%\linenumbers\relax % Commence numbering lines

%\usepackage[showframe,%Uncomment any one of the following lines to test 
%%scale=0.7, marginratio={1:1, 2:3}, ignoreall,% default settings
%%text={7in,10in},centering,
%%margin=1.5in,
%%total={6.5in,8.75in}, top=1.2in, left=0.9in, includefoot,
%%height=10in,a5paper,hmargin={3cm,0.8in},
%]{geometry}

\begin{document}


\title{천연 색소의 추출과 무기 안료의 합성 예비보고서}

\author{서울대학교 전기정보공학부 2018-12432 박정현}
 \email{alexist@snu.ac.kr}
\date{\today}% It is always \today, today,
             %  but any date may be explicitly specified

\begin{abstract}
본 실험에서는 유기안료와 무기안료를 이용해 면섬유를 염색하고 각 안료에 의해 색깔이 나타나는 이유를 이론적으로 이해한다. 또한 유기안료를 이용해 염색한 후 매염을 하여 염료의 화학반응에 따라 색깔이 어떻게 변하는지 확이하고 이를 통해 매염제를 통해 안정도가 어떻게 변화하는지 이해한다.
\end{abstract}

%\keywords{Suggested keywords}%Use showkeys class option if keyword
                              %display desired
\maketitle

%\tableofcontents

\section{\label{sec:level1}Introudction}
\section{\label{sec:level2}염색}
케르세틴은 양파를 노랗게 만드는 물질이다.[1] 케르세틴의 분자식은 아래와 같다. 아래의 구조에서 공유전자쌍이 번갈아 가며 공명하게 되고 해당 공명과 일치하는 파장의 빛이 방출된다. 이러한 빛이 가시광선에 해당하면 우리의 눈에서는 색깔이 있는것으로 인식한다. 반면에 무기 염료의 경우 전이금속에 해당하는 원자의 d level에서 전자들이 전이하면서 빛이 방출된다. 마찬가지로 이러한 빛이 가시광선의 파장대역에 해당하면 눈에서는 색깔이 있는 것으로 인식한다.[2]

매염을 하는 경우 금속 원자와 유기염료 분자와 배위화합물을 형성하게 된다. 이 때 매염제의 경우 $Al_{2}(SO_{4})_{3}\cdot18H_{2}O$, $FeCL_{2}\cdot nH_{2}O$와 같은 물질을 사용하며 금속이온의 리간드에 의해 발생한다. 이러한 배위화합물은 면섬유(셀룰로우스)와 염료의 결합을 단단히 유지할 수 있게 만들어준다.

\section{\label{sec:level1}Experimental}
\section{\label{sec:level2}천연 색소의 추출}
$100mL$ 비커, 열 교반기, 핀셋, 약수저, 종이 타월, 백반($KAI(SO_{4})_{2}\cdot 12H_{2}O$), $FeCL_{2}\cdot nH_{2}O$, $NaHCO_{3}$ 용액 ($0.2g/10mL$), 아세트산, 치자, 소목을 준비한다. 두개의 $100mL$ 비커에 $50mL$씩 물을 채운뒤 각각에 $Al_{2}(SO_{4})_{3}\cdot18H_{2}O$, $FeCL_{2}\cdot nH_{2}O$을 $1.5g$ 씩 넣는다. 이 때 각각에 표시를 한뒤 저어주면서 5분 정도 가열한다. 이후에 면섬유를 각각의 비커에 넣고 2분 가량 가열한 뒤 잘 보관한다. 이후에 $100mL$ 비커 두개에 $50mL$을 물을 가열한 뒤 한쪽에는 양파의 겉껍질을 넣고 다른 비커에는 흰껍질을 잘게 잘라 넣은 뒤 5분 가량 가열한다. 이 때 색이 색이 우려 나온 이후 건더기를 걸러 염색시 얼룩이 생기지 않도록 한다. 이후에 면섬유 두개를 흐르는 물에 적셔준뒤 두 면섬유를 각각 추출된 염액에 넣은 뒤 3분가량 가열한다. 이후에 면섬유를 종이 타월에 올려 놓은 뒤 표시한다.  사용한 염액을 $100mL$ 비커 두개에 각각 분리한다. 이후에는 $Al_{2}(SO_{4})_{3}\cdot18H_{2}O$ 용액에 담가뒀던 면섬유를 각각의 염액에 넣은 뒤 3분 가량 가열한다. 이후에 흐르는 물에 씻은 뒤 표시하고 종이타월에 올려둔다. 마찬가지로 $FeCL_{2}\cdot nH_{2}O$에 담가 가열했던 면섬유를 염액에 넣고 3분 가량 가열한다. 이후에 흐르는 물에 씻고 종이타월에 올린 뒤 표시한다.

만들어진 여 섯개의 면섬유에 아세트산을 한 두 방울 떨어뜨리고, 나머지 부분에 $NaHCO_{3}$ 수용액을 한두 방울 떨어뜨린다. 이후에 흐르는 물에 씻기고 변화를 확인한다.

\section{\label{sec:level2}무기 안료의 합성}
열교반기, 청량 종이, 여과지, 뷰르너 깔때기, 뷰르너 플라스크, 유리막대, 카세인, 그리고 아래 표의 물질들을 준비한다. 두 개의 시험관에 뜨거운 물을 $50mL$가량 채운 뒤 각 색깔에 해당하는 물질을 섞으면 침전이 생긴다. 뷰흐너 깔때기를 이용해 침전을 걸러서 말리고 $100mL$ 비커에 카세인을 넣은 뒤 물을 넣어 걸쭉한 반죽을 만든다. 이후에 해당 반죽을 같은 양의 색소를 넣은 뒤 넣고 잘 저어준다.
\begin{table}[]
\begin{tabular}{c|c|c} \hline \hline
반응 물질 & 침전 물질 & 침전물 색깔 \\ \hline
$0.3g K_{4}Fe(CN)_{6}$ , $0.2g CoCl_{2}$ & $CoFe(CN)_{6}$ & 회녹색 \\ \hline
$0.2g NH_{4}Fe(SO_{4})_{2}$ , $0.2g Na_{2}CO_{3}$ & $Fe(OH)_{3}$ & 갈색 \\ \hline
$0.2g NH_{4}Fe(SO_{4})_{2}$ , $0.2g K_{4}Fe(CN)_{6}$ & $KFe(CN)_{6}$ & 파란색 \\ \hline
$0.2g CoCl_{2}$ , $1.0mL Na_{2}SiO_{3}$ & $CoSiO_{3}$ & 보라색 \\ \hline
$0.2g CoCl_{2}$ , $0.2g Na_{2}CO_{3}$ & $CoCO_{3}$ & 연보라색 \\ \hline \hline
\end{tabular}
\caption{\label{tab:C18}무기 안료를 만들기 위한 물질들과 색깔들}
\end{table}

\section{\label{sec:level1}Reference}
[1] 김희준, \textit{일반화학 실험}(자유아카데미, 2016)\\

[2] D.W. Oxtoby, H.P. Gillis, and L. Butler, \textit{Principles of Modern Chemistry} (Brooks/Cole, Australia, 2020).


\end{document}
%
% ****** End of file apssamp.tex ******
