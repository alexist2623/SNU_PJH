% ****** Start of file apssamp.tex ******
%
%   This file is part of the APS files in the REVTeX 4.2 distribution.
%   Version 4.2a of REVTeX, December 2014
%
%   Copyright (c) 2014 The American Physical Society.
%
%   See the REVTeX 4 README file for restrictions and more information.
%
% TeX'ing this file requires that you have AMS-LaTeX 2.0 installed
% as well as the rest of the prerequisites for REVTeX 4.2
%
% See the REVTeX 4 README file
% It also requires running BibTeX. The commands are as follows:
%
%  1)  latex apssamp.tex
%  2)  bibtex apssamp
%  3)  latex apssamp.tex
%  4)  latex apssamp.tex
%
\documentclass[%
 reprint,
%superscriptaddress,
%groupedaddress,
%unsortedaddress,
%runinaddress,
%frontmatterverbose, 
%preprint,
%preprintnumbers,
%nofootinbib,
%nobibnotes,
%bibnotes,
 amsmath,amssymb,
 aps,
%pra,
%prb,
%rmp,
%prstab,
%prstper,
%floatfix,
]{revtex4-2}
\usepackage{kotex}
\usepackage{graphicx}% Include figure files
\usepackage{dcolumn}% Align table columns on decimal point
\usepackage{bm}% bold math
\usepackage{chemformula}
\usepackage{chemfig}
\usepackage{lewis}
%\usepackage{hyperref}% add hypertext capabilities
%\usepackage[mathlines]{lineno}% Enable numbering of text and display math
%\linenumbers\relax % Commence numbering lines

%\usepackage[showframe,%Uncomment any one of the following lines to test 
%%scale=0.7, marginratio={1:1, 2:3}, ignoreall,% default settings
%%text={7in,10in},centering,
%%margin=1.5in,
%%total={6.5in,8.75in}, top=1.2in, left=0.9in, includefoot,
%%height=10in,a5paper,hmargin={3cm,0.8in},
%]{geometry}

\begin{document}


\title{계산화학2 예비보고서}

\author{서울대학교 전기정보공학부 2018-12432 박정현}
 \email{alexist@snu.ac.kr}
\date{실험일자: 11/14/2023}% It is always \today, today,
             %  but any date may be explicitly specified

\begin{abstract}
본 실험에서는 하트리폭 방법을 이용해 분자들의 최저에너지를 구하는 방법을 이론적으로 이해하고 이것을 직접 프로그램을 이용해 계산한다. 이를 통해 분자들의 구조를 양자역학적으로 이해하고 이를 계산하는 방법을 익힌다. 
\end{abstract}
%\keywords{Suggested keywords}%Use showkeys class option if keyword
                              %display desired
\maketitle
.\newpage
.\newpage

\section{\label{sec:level1}Introudction}
\subsection{\label{sec:level2}실험 배경 및 목적}
물질을 구성하는 분자들은 여러가지의 결합 방식을 통해 원자들이 결합하여 구성된다. 이러한 화학 결합을 이해하기 위해 VSEPR, valence bond theory와 같은 여러가지 방법들이 제시되었다. 양자역학이 개발되면서 분자를 구성하는 원자들을 양자역학을 통해 정확히 기술해야 함을 알게 되면서 복잡한 방정식을 풀기 위해 여러가지의 방법들이 제시되었다. 분자구조를 양자역학적으로 풀어내는 것은 매우 어려운 일이지만 화학 분야에서 매우 중요한 일이다. 따라서 본 실험에서는 분자 구조를 계산하는 하트리 폭 방법에 대해 이해하고 이를 실습을 통해 구현하면서 이에 대한 이해도를 높인다.

\subsection{\label{sec:level2}Hartree Fock Method}
수소의 슈뢰딩거 방정식과 달리 일반적인 원자, 혹은 분자들의 슈뢰딩거 방정식은 전자들 간의 반발, 그리고 각 전자들과 원자 핵 사이에서의 상호작용을 모두 고려하여야 한다. n번째 전자의 파동함수를 $\psi_{n}$이라 하고 원자핵의 질량이 충분히 커 거의 움직이지 않아 Born-Oppenheimer 근사를 적용하여 전자의 파동함수와 원자핵의 파동함수가 $\psi = \psi_{e}\psi_{n}$와 같이 변수 분리된다고 가정할 수 있다. 전자들간의 상호작용을 제외한 각 전자들의 해밀토니안은 아래와 같이 표현된다.
\begin{align}
	h_{i} &= -\frac{\hbar^{2}}{2m}\nabla^{2} - \frac{e^{2}}{4\pi\varepsilon_{0}}\frac{1}{r_{i}}
\end{align}
전자들은 Fermion이므로 동일한 위치에 동일한 양자상태로 존재할 수 없다. 이를 만족하도록 만드는 것이 slater 행렬이며 각 전자들의 파동함수를 $\phi_{i}$로 나타내는 경우 전체 파동함수는 $\Phi = \frac{1}{\sqrt{n!}}\det{\phi_{1}\cdots\phi_{n}}$이다. 전자들은 모두 동일한 입자이고 구분 불가능하므로 전체 에너지는 아래와 같이 나타난다.

\begin{align}
	E &= \sum_{i} [\phi_{i}|h|\phi_{i}]+\frac{1}{2}\sum_{ij}\{[\phi_{i}\phi_{i}|\phi_{j}\phi_{j}]-[\phi_{i}\phi_{j}|\phi_{j}\phi_{i}]\}
\end{align}
\begin{align}
	[\phi_{i}\phi_{j}|\phi_{k}\phi_{l}] &= \int \phi_{i}^{*}(x_{1})\phi_{j}^{*}(x_{1})\frac{e^{2}}{4\pi\varepsilon_{0}r_{12}}\phi_{k}(x_{2})\phi_{l}(x_{2})dx_{1}dx_{2}
\end{align}

변분원리에 의해 임의의 파동함수를 통해 계산한 에너지값은 바닥상태의 에너지보다 큰 값을 가진다.[1] 따라서 규격화 조건을 이용해 라그랑주 곱수법을 이용하면 각각의 파동함수 $\phi_{i}$는 아래의 식을 만족한다.
\begin{align}
	f_{1}\phi_{i} = \epsilon_{i}\phi_{i}
\end{align}
\begin{align}
	f_{1} = h_{1} + \sum_{u}\{J_{u}-K_{u}\}
\end{align}
\begin{align}
	J_{u}\phi_{i}(x_{1}) = \frac{e^{2}}{4\pi\varepsilon_{0}}\left\{ \int \phi_{u}^{*}(x_{2})\frac{1}{r_{12}}\phi_{u}(x_{2})dx_{2} \right\}\phi_{i}(x_{1})
\end{align}
\begin{align}
	K_{u}\phi_{i}(x_{1}) = \frac{e^{2}}{4\pi\varepsilon_{0}}\left\{ \int \phi_{u}^{*}(x_{2})\frac{1}{r_{12}}\phi_{i}(x_{2})dx_{2} \right\}\phi_{u}(x_{1})
\end{align}

이러한 구조의 식을 반복해서 풀어 문제를 푸는 것을 Self Consistent Field방식이라고 한다.[2] 하지만 이러한 2계 미분방정식을 매번 푸는 것은 매우 복잡하고 오래걸린다. 따라서 파동함수를 기저함수 $\theta$로 둔 뒤 기저함수에 곱해지는 계수들을 계산하게 된다. 이때 파동함수는 $\psi_{i} = \sum c_{ji}\theta_{j}$으로 나타나고 아래의 식을 만족한다.

\begin{align}
	f_{1}\sum_{j} c_{ji}\theta_{j} = \epsilon_{i}\sum c_{ji}\theta_{j}
\end{align}

양변을 $\theta_{k}$로 적분하면 아래의 식을 만족한다.

\begin{align}
	\sum_{j} c_{ji}\int \theta_{k}f_{1}\theta_{j}dx_{1} &= \epsilon_{i}\sum_{j} c_{ji}\int \theta_{k}\theta_{j}dx_{1}\\
	\sum_{j} F_{kj}c_{ki}&= \epsilon_{i}\sum_{j} S_{kj}c_{ki}
\end{align}

여기서 F와 S는 각각 Fock matrix, overlap matrix라고 하며 이를 Roothaan equation이라고 하며 더 간략하게는 아래와 같이 나타난다.[3]
\begin{align}
	Fc = Sc\epsilon
\end{align}
따라서 Roothaan equation을 이용해 반복하여 계수 $c_{ij}$를 계산하면 실제 파동함수 값을 빠르게 근사해 나갈 수 있다. 여기서 사용하는 기저함수로는 여러가지가 존재하는데 gaussian 함수가 대표적이며 사용되는 가우시안 함수의 수 $N$에 따라 STO-NG으로 나타낸다.[4]


\section{\label{sec:level1}Assignment}
\subsection{\label{sec:level2}화학 결합 및 에너지}
화학 길이 및 결합 에너지는 Tab.\ref{tab:bond}와 같다.[6][7]

\begin{table}[h]
\caption{\label{tab:bond} 화학 결합 및 에너지}
\begin{tabular}{l|c|c} \hline \hline
결합 & 에너지[kJ/mol] & 결합 길이[\r{A}] \\ \hline
\ch{C-H} &413 &1.10 \\ 
\ch{O-H} &467 &0.96 \\ 
\ch{N-H} &391 &1.01 \\ \hline\hline
\end{tabular}
\end{table}

\subsection{\label{sec:level2}결합각 경향}
\subsubsection{\label{sec:level2}VSEPR}
\ch{H2O}, \ch{NH3}, \ch{CH4}의 Steric Number(SN)은 각각 아래 분자 구조와 같이 모두 4에 해당한다. 따라서 공유 전자쌍과 비공유 전자쌍을 모두 포함하여 정사면체의 분자모양을 띠게 된다. \ch{CH4}의 경우 대칭적이므로 $109.5^{o}$의 각을 이룬다. \ch{H2O}의 경우 lone pair가 두개로 비공유 전자쌍에 의한 반발력이 강해 $104.5^{o}$의 각을 이룬다. \ch{NH3}의 경우 lone pair가 하나로 \ch{H2O}만큼 반발이 강하지 않아 $107.3^{o}$의 각을 이룬다.[4]
\begin{align}
	\schemestart
		\chemfig{O(-[:270]H)(-[:180]:)(-[:90]H)(-[:0]:)}
	\schemestop
\end{align}

\begin{align}
	\schemestart
		\chemfig{N(-[:270]H)(-[:180]H)(-[:90]H)(-[:0]:)}
	\schemestop
\end{align}

\begin{align}
	\schemestart
		\chemfig{C(-[:270]H)(-[:180]H)(-[:90]H)(-[:0]H)}
	\schemestop
\end{align}
\subsubsection{\label{sec:level2}Valence Bond Thoery}
탄소의 electron configuration은 $(1s^{2})(2s^{2})(2p^{2})$이다. 여기서 2s오비탈이 2p오비탈로 promotion되면서 $sp^{3}$혼성 궤도를 가지게 된다. 이를 통해 \ch{H}와 혼성궤도가 모두 sigma 결합을 하게 된다. 따라서 대칭적인 사면체를 이루고 $109.5^{o}$의 각을 이루게된다.

질소의 경우 $(1s^{2})(2s^{2})(2p^{3})$에서 $(2s^{2})(2p^{3})$오비탈이 $sp^{3}$혼성궤도를 이루게 되고 하나의 전자스핀을 가진 오비탈이 \ch{H} sigma 결합을 하게 되고 하나의 lone pair전자들을 포함하여 사면체를 이루게 된다. 여기서 전자들의 반발력에 의해 결합각이 $109.5^{o}$보다 작은 $107.3^{o}$를 이루게 된다.

산소의 경우 $(1s^{2})(2s^{2})(2p^{4})$에서 $(2s^{2})(2p^{4})$오비탈이 $sp^{3}$혼성궤도를 이루게 되고 하나의 전자스핀을 가진 오비탈이 \ch{H} sigma 결합을 하게 되고 2개의  lone pair전자들을 포함하여 사면체를 이루게 된다. 여기서 전자들의 반발력에 의해 결합각이 $109.5^{o}$보다 작은 $104.5^{o}$를 이루게 된다.[5]


\subsection{\label{sec:level2}Experimental}
Edison 플랫폼에 설치되어 있는 ORCA 패키지를 이용해 각 분자들의 최저 결합에너지와 결합 길이를 구한다. 이를 위해 Avogadro 프로그램을 설치한다. 이후에는 EDISON에 설치되어 아후에는 ORCA에 UHF:HF 옵션을 입력하여 하트리폭을 통해 최저 에너지를 계산하도록 한다. 이 대 모든 전자들이 짝을 이루는 경우 HF, 아닌 경우 UHF 키워드를 입력하여 에너지를 계산한다. Energy, SP를 입력하면 에너지를, Opt를 입력하면 최적 구조를, Freq를 입력하면 분자의 진동수를 계산한다. \ch{H2O}, \ch{NH3}, \ch{CH4}에 대한 입력 파일을 만들고 Opt를 입력해 구조 최적화를 수행한다. 이후에 Avogadro를 이용해 출력 결과를 시각화한다. 마지막에는 출력파일에 저장된 에너지 정보를 저장한다.

실습은 개인 실습으로 진행되기 때문에 매뉴얼을 정확하게 숙지하도록 한다. 도한 이론적인 배경 또한 정확히 숙지하도록 한다. 그리고 실습시간에는 컴퓨터로 수업과 관련없는 행동을 하지 않도록 한다.

\section{\label{sec:level1}Reference}
[1] Griffiths, D. J. (2017, January 1). The Variational Principle. In \textit{Introduction to Quantum Mechanics} (2nd ed., pp. 293–294). Cambridge University Press.

[2] Atkins, P. W., \& Friedman, R. S. (2011, January 1). RESTRICTED AND UNRESTRICTED HARTREE–FOCK CALCULATIONS. In \textit{Molecular Quantum Mechanics} (4th ed., pp. 289–291). Oxford University Press.

[3] Atkins, P. W., \& Friedman, R. S. (2011, January 1). RESTRICTED AND UNRESTRICTED HARTREE–FOCK CALCULATIONS. In \textit{Molecular Quantum Mechanics} (4th ed., pp. 293-295). Oxford University Press.

[3] Atkins, P. W., \& Friedman, R. S. (2011, January 1). RESTRICTED AND UNRESTRICTED HARTREE–FOCK CALCULATIONS. In \textit{Molecular Quantum Mechanics} (4th ed., pp. 296-301). Oxford University Press.

[4] Oxtoby, D., Gillis, H., \& Campion, A. (2007, April 2). Chemical Bonding and Molecular Structure. In \textit{Principles of Modern Chemistry} (6th ed., pp. 92-97). Cengage Learning.

[5] Oxtoby, D., Gillis, H., \& Campion, A. (2007, April 2). Chemical Bonding and Molecular Structure. In \textit{Principles of Modern Chemistry} (6th ed., pp. 256-260). Cengage Learning.

[6] Zumdahl, S. S., \& Zumdahl, S. A. (2006, January 9). Covalent Bond Energies and Chemical Reactions. In Chemistry (7th ed., p. 351). Cengage Learning.

[7] Oxtoby, D., Gillis, H., \& Campion, A. (2007, April 2). Chemical Bonding: The Classical Description. In \textit{Principles of Modern Chemistry} (6th ed., p. 82). Cengage Learning.


\end{document}
%
% ****** End of file apssamp.tex ******
