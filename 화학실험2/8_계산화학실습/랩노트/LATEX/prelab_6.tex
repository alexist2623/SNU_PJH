% ****** Start of file apssamp.tex ******
%
%   This file is part of the APS files in the REVTeX 4.2 distribution.
%   Version 4.2a of REVTeX, December 2014
%
%   Copyright (c) 2014 The American Physical Society.
%
%   See the REVTeX 4 README file for restrictions and more information.
%
% TeX'ing this file requires that you have AMS-LaTeX 2.0 installed
% as well as the rest of the prerequisites for REVTeX 4.2
%
% See the REVTeX 4 README file
% It also requires running BibTeX. The commands are as follows:
%
%  1)  latex apssamp.tex
%  2)  bibtex apssamp
%  3)  latex apssamp.tex
%  4)  latex apssamp.tex
%
\documentclass[%
 reprint,
%superscriptaddress,
%groupedaddress,
%unsortedaddress,
%runinaddress,
%frontmatterverbose, 
%preprint,
%preprintnumbers,
%nofootinbib,
%nobibnotes,
%bibnotes,
 amsmath,amssymb,
 aps,
%pra,
%prb,
%rmp,
%prstab,
%prstper,
%floatfix,
]{revtex4-2}
\usepackage{kotex}
\usepackage{graphicx}% Include figure files
\usepackage{dcolumn}% Align table columns on decimal point
\usepackage{bm}% bold math
\usepackage{chemformula}
\usepackage{chemfig}
\usepackage{lewis}
%\usepackage{hyperref}% add hypertext capabilities
%\usepackage[mathlines]{lineno}% Enable numbering of text and display math
%\linenumbers\relax % Commence numbering lines

%\usepackage[showframe,%Uncomment any one of the following lines to test 
%%scale=0.7, marginratio={1:1, 2:3}, ignoreall,% default settings
%%text={7in,10in},centering,
%%margin=1.5in,
%%total={6.5in,8.75in}, top=1.2in, left=0.9in, includefoot,
%%height=10in,a5paper,hmargin={3cm,0.8in},
%]{geometry}

\begin{document}


\title{계산화학2 랩노트}

\author{서울대학교 전기정보공학부 2018-12432 박정현}
 \email{alexist@snu.ac.kr}
\date{실험일자: 11/14/2023}% It is always \today, today,
             %  but any date may be explicitly specified

\maketitle

\section{\label{sec:level1}실험 과정}
EDISON에 접속한 뒤 C212512432를 입력하고 비밀번호는 아이디와 동일하게 입력한다. ORCA를 검색한 뒤 RUN을 눌러 ORCA를 실행한다. 그 뒤 Avogadro 프로그램을 실행한뒤 \ch{CH4}는 Element에서 Carbon을, \ch{NH3}, \ch{H2O}는 각각 질소, 산소를 선택한다. 그 뒤 화면의 아무 부분을 클릭하여 분자를 배치한다. 그 뒤 Extensions에서 Orca에서 Generate Orca Input file을 선택하여 input file을 만든다. 이 때 Calculation은 Geometry Optimization으로 바꾸고 아래의 ! 코드에 ! HF OPT 6-31g(d) LargePrint를 입력하여 하트리 폭 방식으로 가우시안 기저를 통해 문제를 풀도록 설정한다. 그 뒤 input file을 Orca에서 Menu를 눌러 입력한 뒤 Submit을 눌러 파일을 제출한다. 5분 가량 기다린 뒤 결과가 나타나면 Download에서 .out파일을 저장한다. 이후에 .out파일을 Avogadro를 통해 연 뒤 View, Properties에서 각각의 계산값을 확인한다.
\end{document}
%
% ****** End of file apssamp.tex ******
