% ****** Start of file apssamp.tex ******
%
%   This file is part of the APS files in the REVTeX 4.2 distribution.
%   Version 4.2a of REVTeX, December 2014
%
%   Copyright (c) 2014 The American Physical Society.
%
%   See the REVTeX 4 README file for restrictions and more information.
%
% TeX'ing this file requires that you have AMS-LaTeX 2.0 installed
% as well as the rest of the prerequisites for REVTeX 4.2
%
% See the REVTeX 4 README file
% It also requires running BibTeX. The commands are as follows:
%
%  1)  latex apssamp.tex
%  2)  bibtex apssamp
%  3)  latex apssamp.tex
%  4)  latex apssamp.tex
%
\documentclass[%
 reprint,
%superscriptaddress,
%groupedaddress,
%unsortedaddress,
%runinaddress,
%frontmatterverbose, 
%preprint,
%preprintnumbers,
%nofootinbib,
%nobibnotes,
%bibnotes,
 amsmath,amssymb,
 aps,
%pra,
%prb,
%rmp,
%prstab,
%prstper,
%floatfix,
]{revtex4-2}
\usepackage{kotex}
\usepackage{graphicx}% Include figure files
\usepackage{dcolumn}% Align table columns on decimal point
\usepackage{bm}% bold math
\usepackage{chemformula}
\usepackage{chemfig}
%\usepackage{hyperref}% add hypertext capabilities
%\usepackage[mathlines]{lineno}% Enable numbering of text and display math
%\linenumbers\relax % Commence numbering lines

%\usepackage[showframe,%Uncomment any one of the following lines to test 
%%scale=0.7, marginratio={1:1, 2:3}, ignoreall,% default settings
%%text={7in,10in},centering,
%%margin=1.5in,
%%total={6.5in,8.75in}, top=1.2in, left=0.9in, includefoot,
%%height=10in,a5paper,hmargin={3cm,0.8in},
%]{geometry}

\begin{document}


\title{수소이야기}

\author{서울대학교 전기정보공학부 2018-12432 박정현}
 \email{alexist@snu.ac.kr}
\date{실험일자: 11/7/2023}% It is always \today, today,
             %  but any date may be explicitly specified

\begin{abstract}
본 실험에서는 염산과 금속을 반응시켜 수소 기체를 형성하고 연소시킴으로서 해당 기체가 수소임을 확인한다. 그리고 각 금속과 염산을 반응시켜 수소기체의 부피를 측정해 각 금속의 당량을 확인한다. 그리고 물을 전기분해 시킴으로서 물의 발견 과정에 대해서 이해한다. 마지막으로는 수소의 스펙트럼을 측정함으로써 수소의 양자역학적인 구조에 대해 이해도를 높인다.
\end{abstract}

%\keywords{Suggested keywords}%Use showkeys class option if keyword
                              %display desired
\maketitle

%\tableofcontents

\section{\label{sec:level1}Introudction}
\subsection{\label{sec:level2}실험 배경 및 목적}
 수소는 물질을 이루는 가장 기본적인 원소이자 우주에서 가장 오래되고 많은 양을 차지하고 있는 원소이다. 또한 생물체에서의 화학반응에서 빠지지 않고 등장하는 원소이다. 따라서 수소에 대한 이해도를 높이는 것은 매우 중요한 일이다. 본 실험에서는 수소가 어떤 방법을 통해 발견되었는지 실험적으로 재현하고 금속의 화학반응을 이용해 금속의 당량을 측정하여 수소의 발견 과정에 대한 이해도를 높인다. 그리고 수소와 산소 사이의 전기음성도 차이를 물의 전기분해 실험을 통해 이해한다. 마지막으로는 수소의 선스펙트럼을 측정하여 수소를 양자역학적인 관점에서 이해한다.[1]

\subsection{\label{sec:level2}수소의 발생 및 폭명성 실험}
아연과 염산의 화학 반응은 아래와 같다. 따라서 화학반응을 통해 수소 기체가 형성됨을 알 수 있다.
\begin{align}
	\ch{Zn(s) + 2HCl(aq) ->ZnCl2(aq) + H2(g)}
\end{align}

\subsection{\label{sec:level2}금속 원소의 당량 결정}
마그네슘, 알루미늄, 스칸듐, 아연과 \ch{HCl}과의 화학반응은 아래와 같다. 아래의 화학반응식을 통해 각 원소들이 아래의 화학반응식을 따라 반응하는지 확인할 수 있다.
\begin{align}
	\ch{Zn(s) + 2HCl(aq) ->ZnCl2(aq) + H2(g)}
\end{align}
\begin{align}
	\ch{Mg(s) + 2HCl(aq) ->MgCl2(aq) + H2(g)}
\end{align}
\begin{align}
	\ch{2Sc(s) + 6HCl(aq) ->2ZnCl3(aq) + 3H2(g)}
\end{align}
\begin{align}
	\ch{2Al(s) + 6HCl(aq) ->2AlCl3(aq) + 3H2(g)}
\end{align}

\subsection{\label{sec:level2}물의 전기분해}
Cathod와 anode에서 나타나는 화학반응식은 각각 아래와 같다.
\begin{align}
	\ch{4 H+ (aq) + 4 e- -> 2 H2 (g)}
\end{align}
\begin{align}
	\ch{2 H2O (l) -> O2 (g) + 4 H+ (aq)}
\end{align}

따라서 알짜 화학반응식은 아래와 같다.
\begin{align}
	\ch{2 H2O (l) -> 2 H2 (g) + O2 (g)}
\end{align}
이를 통해 생성된 기체의 비를 확인하면 실제 화학반응이 위에서 예측한 화학반응식으로 생성된 것인지 확인할 수 있다.[3]

\subsection{\label{sec:level2}수소의 선스펙트럼}
수소의 양성자와 전자는 $-\frac{e^{2}}{4\pi \epsilon r}$의 퍼텐셜 에너지를 가진다. 이를 통해 슈뢰딩거 방정식을 풀면 아래와 같은 에너지 준위를 가짐을 알 수 있다.[2]
\begin{align}
	E_{n} &= -\frac{R}{n^{2}}
\end{align}
여기서 R은 리드베리 상수이다. 따라서 방출, 흡수되는 광자의 주파수는 아래의 관계식을 가진다. 아래의 관계식을 이용하면 리드베리 상수를 측정할수 있다.
\begin{align}
	hf &= \left( \frac{1}{n^{2}} - \frac{1}{m^{2}} \right)
\end{align}

\section{\label{sec:level1}Experimental}
\subsection{\label{sec:level2}수소의 발생 및 폭명성 실험}
6N 염산, 아연, 가지가 달린 플라스크, 고무관, 빨대, 비눗물 그리고 라이터를 준비한다. 가지 플라스크에 고무관과 빨대를 연결한다. 플라스크에 아연조각을 넣고 6N 염산을 넣는다. 빨대 끝에 비눗방울이 생기긱 시작하면 비눗방울을 공중으로 분리시킨 후 라이터를 통해 연소시켜 해당 물질이 수소임을 확인한다.

\subsection{\label{sec:level2}금속 원소의 당량 결정}
6N 염산, 마그네슘, 알루미늄, 스칸듐, 아연과 같은 금속, 100mL의 가지가 달린 플라스크, 호스, 자석 젓개와 막대, 1L 비커, 100mL 메스실린더, 클램프, 그리고 금속을 갈기 위한 사포와 주사기를 준비한다. 비커에 물을 넣고 물이 가득찬 메스실린더를 뒤집어 넣어 공기가 없도록 만든다. 금속을 사포로 갈아 40mg을 가지 플라스크에 넣고 금속이 약간 잠긴 상태가 되도록 주사기로 6N HCl을 약 1~2mL를 가한다. 이후에 자석 젓개를 통해 기체를 포집한 후 금속이 나타나지 않으면 메스실린더에 모인 기체의 양을 측정한다. 동일한 실험을 앞서 준비한 모든 금속에서 반복실험 한다.

\subsection{\label{sec:level2}물의 전기분해}
1L의 비커, 백금, 페트리 접시, 전선, 전원 장치, 메스실린더, 스탠드, 그리고 0.1M \ch{H2SO4}를 준비한다. 전해질로 0.1M 황산으로 구성된 전기분해 장치를 조립하고 각각의 전극에서 나타나는 화학 반응을 관찰한다.

\subsection{\label{sec:level2}수소의 선스펙트럼}
수소 방전관과 분광기를 준비한다. 이후에 방전관에서 나타난 빛을 분광기를 통해 측정하고 수소의 스펙트럼을 측정하고 기록한다.

\subsection{\label{sec:level2}유의 사항}
수소 발생 실험에서 수소가 플랗스크 채로 폭발할 수 있으니 꼭 플라스크를 비눗방울과 분리시켜 공중으로 띄운뒤 연소시켜야 함에 주의한다. 금속 원소로부터 기체를 포집할 때 메스실린더에서 물리 쏟아질 수 있으니 주의한다. 물의 전기분해 실험에서 황산은 강산이니 취급에 주의하도록 한다.


\section{\label{sec:level1}Reference}
[1] 김. (2010, August 1). 수소이야기. In \textit{일반화학실험} (1th ed., p.p. 167-168).

[2] Oxtoby, D., Gillis, H., \& Campion, A. (2007, April 2). Quantum Mechanics and Atomic Structure. In \textit{Principles of Modern Chemistry} (6th ed., pp. 176-179). Cengage Learning.

[3] Oxtoby, D., Gillis, H., \& Campion, A. (2007, April 2). Electrochemistry. In \textit{Principles of Modern Chemistry} (6th ed., pp. 744-745). Cengage Learning.


\end{document}
%
% ****** End of file apssamp.tex ******
