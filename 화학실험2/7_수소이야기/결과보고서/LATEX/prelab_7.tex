% ****** Start of file apssamp.tex ******
%
%   This file is part of the APS files in the REVTeX 4.2 distribution.
%   Version 4.2a of REVTeX, December 2014
%
%   Copyright (c) 2014 The American Physical Society.
%
%   See the REVTeX 4 README file for restrictions and more information.
%
% TeX'ing this file requires that you have AMS-LaTeX 2.0 installed
% as well as the rest of the prerequisites for REVTeX 4.2
%
% See the REVTeX 4 README file
% It also requires running BibTeX. The commands are as follows:
%
%  1)  latex apssamp.tex
%  2)  bibtex apssamp
%  3)  latex apssamp.tex
%  4)  latex apssamp.tex
%
\documentclass[%
 reprint,
%superscriptaddress,
%groupedaddress,
%unsortedaddress,
%runinaddress,
%frontmatterverbose, 
%preprint,
%preprintnumbers,
%nofootinbib,
%nobibnotes,
%bibnotes,
 amsmath,amssymb,
 aps,
%pra,
%prb,
%rmp,
%prstab,
%prstper,
%floatfix,
]{revtex4-2}
\usepackage{kotex}
\usepackage{graphicx}% Include figure files
\usepackage{dcolumn}% Align table columns on decimal point
\usepackage{bm}% bold math
\usepackage{chemformula}
\usepackage{chemfig}
%\usepackage{hyperref}% add hypertext capabilities
%\usepackage[mathlines]{lineno}% Enable numbering of text and display math
%\linenumbers\relax % Commence numbering lines

%\usepackage[showframe,%Uncomment any one of the following lines to test 
%%scale=0.7, marginratio={1:1, 2:3}, ignoreall,% default settings
%%text={7in,10in},centering,
%%margin=1.5in,
%%total={6.5in,8.75in}, top=1.2in, left=0.9in, includefoot,
%%height=10in,a5paper,hmargin={3cm,0.8in},
%]{geometry}

\begin{document}


\title{수소이야기 결과보고서}

\author{서울대학교 전기정보공학부 2018-12432 박정현}
 \email{alexist@snu.ac.kr}
\date{실험일자: 11/7/2023}% It is always \today, today,
             %  but any date may be explicitly specified

\begin{abstract}
본 실험에서는 
\end{abstract}

%\keywords{Suggested keywords}%Use showkeys class option if keyword
                              %display desired
\maketitle

%\tableofcontents

\section{\label{sec:level1}Assignment}
\subsection{\label{sec:level2}1}
\subsubsection{\label{sec:level3}재결정 정제화란}

\section{\label{sec:level1}Data \& Results}
\subsection{\label{sec:level2}수소의 폭명성 실험}
6N 염산과 아연을 반응시킨 기체를 연소한 결과 펑 소리를 내며 폭발하는 것을 확인할 수 있었다. 따라서 생성된 기체는 수소기체일 것으로 결론지었다.

\subsection{\label{sec:level2}금속 원소의 당량 결정}
각 금속과의 반응에서 생성된 기체의 양은 아래와 같다.

\section{\label{sec:level1}Reference}
[1] Oxtoby, D., Gillis, H., \& Campion, A. (2007, April 2). Chemical Equilibrium. In \textit{Principles of Modern Chemistry} (6th ed., pp. 603-605). Cengage Learning.

[2] Atkins, P., Jones, L., \& Laverman, L. (2012, December 21). PHYSICAL EQUILIBRIA. In Chemical Principles (5th ed., pp. 367–368). W. H. Freeman.

[3] Armarego, W., \& Chai, C. (2009, July 23). COMMON PHYSICAL TECHNIQUES USED IN PURIFICATION. In \textit{Purification of Laboratory Chemicals} (5th ed., pp. 14–37). Butterworth-Heinemann.

[4] Armarego, W., \& Chai, C. (2009, July 23). COMMON PHYSICAL TECHNIQUES USED IN PURIFICATION. In \textit{Purification of Laboratory Chemicals} (5th ed., pp. 18–38). Butterworth-Heinemann.

[5] Oxtoby, D., Gillis, H., \& Campion, A. (2007, April 2). Chemical Equilibrium. In \textit{Principles of Modern Chemistry} (6th ed., pp. 602). Cengage Learning.

[6] Smith, J. (2010, January 8). An Introduction to NMR Spectroscopy. In \textit{Organic Chemistry} (3rd ed., pp. 495-517). McGraw-Hill Education.

[7] Smith, J. (2010, January 8). Mass Spectrometry and Infrared Spectroscopy. In \textit{Organic Chemistry} (3rd ed., pp. 464-485). McGraw-Hill Education.

[8] Beiser, A. (1981, January 1). Nuclear Structure. In Concepts of Modern Physics (pp. 394–396). McGraw-Hill Companies.

[9] Haynes, W. M. (2016, April 19). PHYSICAL CONSTANTS OF ORGANIC COMPOUNDS. In CRC Handbook of chemistry and physics (89th ed., p. 127). CRC Press.

\end{document}
%
% ****** End of file apssamp.tex ******
