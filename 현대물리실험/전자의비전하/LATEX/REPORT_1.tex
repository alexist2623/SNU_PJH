% ****** Start of file apssamp.tex ******
%
%   This file is part of the APS files in the REVTeX 4.2 distribution.
%   Version 4.2a of REVTeX, December 2014
%
%   Copyright (c) 2014 The American Physical Society.
%
%   See the REVTeX 4 README file for restrictions and more information.
%
% TeX'ing this file requires that you have AMS-LaTeX 2.0 installed
% as well as the rest of the prerequisites for REVTeX 4.2
%
% See the REVTeX 4 README file
% It also requires running BibTeX. The commands are as follows:
%
%  1)  latex apssamp.tex
%  2)  bibtex apssamp
%  3)  latex apssamp.tex
%  4)  latex apssamp.tex
%
\documentclass[%
 reprint,
%superscriptaddress,
%groupedaddress,
%unsortedaddress,
%runinaddress,
%frontmatterverbose, 
%preprint,
%preprintnumbers,
%nofootinbib,
%nobibnotes,
%bibnotes,
 amsmath,amssymb,
 aps,
%pra,
%prb,
%rmp,
%prstab,
%prstper,
%floatfix,
]{revtex4-2}
\usepackage{kotex}
\usepackage{graphicx}% Include figure files
\usepackage{dcolumn}% Align table columns on decimal point
\usepackage{bm}% bold math
%\usepackage{hyperref}% add hypertext capabilities
%\usepackage[mathlines]{lineno}% Enable numbering of text and display math
%\linenumbers\relax % Commence numbering lines

%\usepackage[showframe,%Uncomment any one of the following lines to test 
%%scale=0.7, marginratio={1:1, 2:3}, ignoreall,% default settings
%%text={7in,10in},centering,
%%margin=1.5in,
%%total={6.5in,8.75in}, top=1.2in, left=0.9in, includefoot,
%%height=10in,a5paper,hmargin={3cm,0.8in},
%]{geometry}

\def\rcurs{{\mbox{$\resizebox{.16in}{.08in}{\includegraphics{ScriptR}}$}}}
\def\brcurs{{\mbox{$\resizebox{.16in}{.08in}{\includegraphics{BoldR}}$}}}
\def\hrcurs{{\mbox{$\hat \brcurs$}}}

\begin{document}


\title{비전하 실험 보고서}

\author{서울대학교 전기정보공학부 2018-12432 박정현}
 \email{alexist@snu.ac.kr}
\date{\today}% It is always \today, today,
             %  but any date may be explicitly specified

\begin{abstract}
본 실험에서는 물질의 특성에 따른 전기전도도를 확인하고, 여러가지의 금속에 대한 전기화학적 서열을 확인한다. 또한 다니엘 전지를 제작한 후 농도에 따른 기전력을 측정해 네른스트 식을 검증하고 이해한다. 화학전지를 이용해 염의 용해도곱 상수를 직접 계산하여 화학전지와 용해도곱상수에 대한 이해도를 높인다. 
\end{abstract}

%\keywords{Suggested keywords}%Use showkeys class option if keyword
                              %display desired
\maketitle

%\tableofcontents

\section{\label{sec:level1}Introudction}
\subsection{\label{sec:level2}Thermonic Emission}
금속에 충분한 열이 가해져 온도가 높아지면 전자가 방출되게 된다. 이러한 현상을 thermonic emission이라고 하며 이 때 방출되는 전류는 금속의 conduction band로부터 금속의 일함수를 넘어 자유전자가 되어 나타나는 전류이다. 이러한 전류는 페르미 분포를 따르는 전자중 충분한 에너지를 가지고 있는 전자가 넘어가는 전류와 터널링 현상을 통해 넘어가는 전류 두 종류가 있으며 아래와 같이 나타난다.[1] 여기서 $A$는 Richard 상수이며 $T$는 온도, 그리고 $\varphi$는 일함수에 해당한다. 충분히 높은 전압에서 가열된 금속의 온도가 높아짐에 따라 방출되는 전류 값이 증가함을 알 수 있다.

\begin{align}
	J &= AT^{2}\exp\left(-\frac{-\varphi}{kT}\right)
\end{align}

\subsection{\label{sec:level2}Helmholtz Coil}
반지름 $R$을 가지는 코일이 중심으로부터 $x$의 거리에 만드는 자기장은 아래와 같다.
\begin{align}
	B_{z} &= \frac{\mu_{0}}{2}\frac{R^{2}I}{\left(x^{2} + R^{2}\right)^{\frac{3}{2}}}
\end{align}
쿨롱 게이지에서 원형코일을 포함한 $xy$평면에서의 vector potential $\vec{A}$는 아래와 같다. 
\begin{align}
	\vec{A}(\vec{r}) &= \frac{\mu_{0}}{4\pi}\int \frac{\vec{J}(\vec{r'})}{\rcurs}d^{3}\vec{r'}
\end{align}
해당 식은 아래와 같이 전개된다. 단, $\rho = r/R$이며 $r$은 중심으로부터 벗어난 거리이다. 그리고 $P_{l}(x)$는 르장드르 다항식에 해당한다.
\begin{align}
	&=\frac{\mu_{0}}{4\pi R}\int \frac{\vec{J}(\vec{r'})}{\sqrt{1+\rho^{2} -2\rho \cos \theta}}d^{3}\vec{r'}\\
	&= \frac{\mu_{0}} {4\pi R}\sum\int \vec{J}(\vec{r'}) \rho^{l} P_{l}(\cos \theta) d^{3}\vec{r'}
\end{align}
$\vec{J}$가 $I/\pi a^{2} \delta(\vec{r} - \vec{r'})$형태를 가지므로 식은 아래와 같아진다.
\begin{align}
	&= \hat{\varphi}\frac{\mu_{0}I}{4\pi }\sum\int_{0}^{2\pi} \rho^{l} P_{l}(\cos \theta) \cos \theta d\theta\\
\end{align}
르장드르 다항식이 짝수차수에서 우함수이므로 적분값이 $0$이된다. 따라서 3차항까지만 고려했을 때 식은 아래와 같다.
\begin{align}
	\vec{A}(\vec{r}) &\simeq \hat{\varphi}\frac{\mu_{0}I}{4\pi }\int_{0}^{2\pi} \rho\cos^{2}\theta +  \rho^{3}\left( \frac{5\cos^{3}\theta -3\cos\theta}{2} \right)\cos\theta  d\theta\\
	&= \hat{\varphi}\frac{\mu_{0}I}{4 }\left( \rho + \frac{3}{8}\rho^{3}\right)\\
	&= \hat{\varphi}\frac{\mu_{0}I}{4 }\left(\frac{r}{R} + \frac{3}{8}\left(\frac{r}{R}\right)^{3}\right)
\end{align}
따라서 원형 코일 근처에서 자기장은 아래와 같아지며 중심으로부터 멀어질수록 자기장의 세기가 강해짐을 알 수 있으나 반지름 $15cm$의 코일에 대해서 $r=3cm$인 경우 약 $0.03$만큼의 오차가 발생하여 해당 항은 큰 기여를 하지 못함을 알 수 있다.
\begin{align}
	B_{z}(r) &\simeq \frac{\mu_{0}I}{2}\left( 1 + \frac{3r^{2}}{4R^{2}} \right)
\end{align}
따라서 거리가 $2x$만큼 떨어진 헬름홀츠 코일 중심 근방 크게 벗어나지 않는 곳에서의 자기장은 아래와 같이 나타난다고 가정하여도 무방하다.
\begin{align}
	B_{z} &\simeq \frac{\mu_{0}NR^{2}I}{\left(x^{2} + R^{2}\right)^{\frac{3}{2}}}
\end{align}

\subsection{\label{sec:level2}Electron in B field}
자기장에서 운동하는 전자의 운동방정식은 아래와 같이 기술된다.
\begin{align}
	m\ddot{\vec{r}} &= e\dot{\vec{r}} \times \vec{B}
\end{align}
$\vec{B} = B_{0} \hat{z}$으로 기술된다고 가정하고 초기 속도가 $(v\cos\theta , v\sin\theta)$로 주어질 때 아래와 같이 운동을 기술할 수 있다.
\begin{align}
	\frac{mv^{2}\cos^{2}\theta}{R} = ev\cos\theta B_{0}
\end{align}
$B_{0} = CI$로 주어지는 경우 식은 아래와 같이 정리할 수 있다.
\begin{align}
	I = \frac{m}{Ce}\frac{v\cos\theta}{R}
\end{align}
$mv^{2}/2 = eV$이므로 해당식은 다시 아래와 같이 정리된다.
\begin{align}
	I^{2} &= \frac{m^{2}}{C^{2}e^{2}}\frac{v^{2}\cos^{2}\theta}{R^{2}}\\
	&= \frac{m}{e}\frac{2V\cos^{2}\theta}{R^{2}}
\end{align}

\section{\label{sec:level1}Reference}
[1] Semiconductor SZE




\end{document}
%
% ****** End of file apssamp.tex ******
