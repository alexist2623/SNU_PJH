\documentclass[a4paper]{oblivoir}

\title{쟤료공학개론 과제2}
\author{2018-12432, Electrical and Computer Engineering department, ParkJeonghyun}
\date{10/3/2023}

\newcommand{\be}{\begin{equation}}
\newcommand{\ee}{\end{equation}}

\usepackage{fapapersize}
\usepackage{amsmath}
\usepackage{MnSymbol}
\usepackage{wasysym}
\usepackage{graphicx}
\usepackage{caption}
\usepackage{subfig}
\usepackage{hyperref}
\usepackage{cite}
\usepackage{dtk-logos}
\usepackage{physics}
\usepackage{tikz}
\usetikzlibrary{decorations.markings, positioning}
\usepackage{dtk-logos}
\usepackage{fancyvrb}
\usepackage{array} 

\usefapapersize{ 210mm, 297mm, 15mm, 15mm, 15mm, 15mm}
\DeclareGraphicsExtensions{.pdf, .png, .jpg}

\renewcommand{\figurename}{Figure}

\begin{document}

\maketitle
\section{Problem 1}
\subsection{A}
A벡터는 아래와 같다.
\begin{align}
	\vec{A} = -\hat{e}_{1}  + \hat{e}_{2} + 0 \hat{e}_{3}
\end{align}
따라서 indice는 아래와 같다.
\begin{align}
	[\bar{1}10]
\end{align}

\subsection{B}
B벡터는 아래와 같다.
\begin{align}
	\vec{B} = \hat{e}_{1}  + 1/2 \hat{e}_{2} + 1/2 \hat{e}_{3}
\end{align}
따라서 indice는 아래와 같다.
\begin{align}
	[211]
\end{align}

\subsection{C}
C벡터는 아래와 같다.
\begin{align}
	\vec{C} = 0\hat{e}_{1}  - 1/2 \hat{e}_{2} - 1 \hat{e}_{3}
\end{align}
따라서 indice는 아래와 같다.
\begin{align}
	[0\bar{1}\bar{2}]
\end{align}

\subsection{D}
D벡터는 아래와 같다.
\begin{align}
	\vec{D} = 1/2\hat{e}_{1}  - 1 \hat{e}_{2} + 1/2 \hat{e}_{3}
\end{align}
따라서 indice는 아래와 같다.
\begin{align}
	[1\bar{2}1]
\end{align}

\section{Problem 2}
\subsection{A}
교차하는 점은 각각 $(1,0,0)$, $(0,1,0)$, $(0,0,-1)$이다. 따라서 miller index는 아래와 같다.
\begin{align}
	(11\bar{1})
\end{align}

\subsection{B}
교차하는 점은 각각 $(1/2,0,0)$, $(0,1/3,0)$, $(0,0,\infty)$이다. 따라서 miller index는 아래와 같다.
\begin{align}
	(320)
\end{align}

\section{Problem 3}
Polycrystalline 물질은 방향성이 grain에 의해 random하게 분포하여 있다. 따라서 이러한 randomness에 의해 pure crystal보다 더 isotropic한 특징을 나타낸다.

\end{document}