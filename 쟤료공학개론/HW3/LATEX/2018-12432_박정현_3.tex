\documentclass[a4paper]{oblivoir}

\title{쟤료공학개론 과제3}
\author{2018-12432, Electrical and Computer Engineering department, ParkJeonghyun}
\date{10/3/2023}

\newcommand{\be}{\begin{equation}}
\newcommand{\ee}{\end{equation}}

\usepackage{fapapersize}
\usepackage{amsmath}
\usepackage{MnSymbol}
\usepackage{wasysym}
\usepackage{graphicx}
\usepackage{caption}
\usepackage{subfig}
\usepackage{hyperref}
\usepackage{cite}
\usepackage{dtk-logos}
\usepackage{physics}
\usepackage{tikz}
\usetikzlibrary{decorations.markings, positioning}
\usepackage{dtk-logos}
\usepackage{fancyvrb}
\usepackage{array} 
\usepackage{chemformula}

\usefapapersize{ 210mm, 297mm, 15mm, 15mm, 15mm, 15mm}
\DeclareGraphicsExtensions{.pdf, .png, .jpg}

\renewcommand{\figurename}{Figure}

\begin{document}

\maketitle
\section{Problem 1}
\subsection{a}
Vacancy의 수는 아래와 같다.
\begin{align}
	N_{v} &= N\exp(-\frac{Q_{v}}{kT})
\end{align}
따라서 fraction은 아래와 같다.
\begin{align}
	N_{v}/N &= \exp(-\frac{0.90 \times 1.602 \times 10^{-19}}{1.38\times 10^{-23}\times 1357})\\
	&= 4.5\times 10^{-4}
\end{align}

\subsection{b}
\begin{align}
	N_{v}/N &= \exp(-\frac{0.90 \times 1.602 \times 10^{-19}}{1.38\times 10^{-23}\times 298})\\
	&= 5.9\times 10^{-16}
\end{align}

\subsection{c}
\begin{align}
	\frac{N_{v}/N(1357K)}{N_{v}/N(298K)} &= \frac{\exp(-\frac{0.90 \times 1.602 \times 10^{-19}}{1.38\times 10^{-23}\times 1357})}{\exp(-\frac{0.90 \times 1.602 \times 10^{-19}}{1.38\times 10^{-23}\times 298})}\\
	&= 7.6\times 10^{11}
\end{align}


\section{Problem 2}
\subsection{a}
Hume – Rothery rule은 1. 원자들의 반지름 크기 차이가 $15\%$ 이하이고, 2. 동일한 결정구조를 가지고, 3. 비슷한 전기음성도를 가지고, 4. 동일한 valency electron을 가져야한다. 이중에 이를 모두 만족하는 원소는 \ch{Pt}뿐이다.

\subsection{b}
\ch{Ag}는 원자반지름 크기가 약 $16\%$차이가 나며 valency 숫자가 다르다. 그리고 \ch{Al}은 원자반지름 크기가 약  $14.8\%$이고 valency 숫자가 다른다. \ch{Co}, \ch{Cr}, \ch{Fe}, \ch{Zn} 는 다른 결정 구조를 가진다. 따라서 \ch{Ag}, \ch{Al}, \ch{Co}, \ch{Cr}, \ch{Fe}, \ch{Zn}는 incomplete solubility를 가진다.

\subsection{c}
\ch{C}, \ch{H}, \ch{O}는 \ch{Ni}보다 크기가 충분히 작으므로 interstitial solid solution이 된다.

\section{Problem 3}
\subsection{a}
Vacancy diffusion은 원래의 lattice position에 존재하던 원자가 근처의 vacant lattice로 이동하는 diffusion이다. 반면에 interstitial diffusion은 격자들 사이의 비어있는 insterstitial position으로 결정을 구성하는 원자보다 작은 분자들이 이동하는 현상이다.

\subsection{b} 
Interstitial diffusion의 경우 결정을 구성하는 원자보다 작은 입자가 이동하는 것이므로 통상적으로 intersitial diffusion이 vacancy diffusion보다 빠르다. 더불어서 vacancy의 숫자보다 interstitial의 숫자가 더 많으므로 (vacancy ~ $\exp(-Q_{v}/kT)$) interstitial diffusion이 발생할 확률이 더 높아 속도가 vacancy diffusion보다 빠르다.

\section{Problem 4}
\subsection{a}
Diffusion constant는 아래의 식을 따른다.
\begin{align}
	D &= D_{0}\exp(-\frac{Q_{d}}{RT})\\
	\ln D &= \ln D_{0} -\frac{Q_{d}}{RT}
\end{align}
따라서 아래의 식이 성립한다.
\begin{align}
	\ln (5.5\times 10^{-14})  - \ln (3.9\times 10^{-13})&=  -\frac{Q_{d}}{R}\left( \frac{1}{600+273.15}  - \frac{1}{700+273.15} \right)\\
	Q_{d} &= -\frac{\ln (5.5\times 10^{-14})  - \ln (3.9\times 10^{-13})}{\left( \frac{1}{600+273.15}  - \frac{1}{700+273.15} \right)}\times 8.3145 [J/mol]\\
	&= 1.38\times 10^{5} [J/mol]
\end{align}
\begin{align}
	D_{0} &= D\times \exp(\frac{Q_{d}}{RT})\\
	&= 5.5\times 10^{-14} \times \exp(\frac{1.38\times 10^{5}}{8.3145\times (273.15+600)}) [m^{2}/s]\\
	&= 9.9 \times 10^{-6} [m^{2}/s]
\end{align}

\subsection{b}
\begin{align}
	D &= D_{0}\exp(-\frac{Q_{d}}{RT})\\
	&= 9.9 \times 10^{-6}\times \exp(-\frac{1.38\times 10^{5}}{8.3145 \times (273.15+850)}) [m^{2}/s]\\
	&= 3.8\times 10^{-12} [m^{2}/s]
\end{align}

\end{document}